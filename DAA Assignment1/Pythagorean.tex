% This is a sample LaTeX input file.  (Version of 12 August 2004.)
%
% A '%' character causes TeX to ignore all remaining text on the line,
% and is used for comments like this one.


\documentclass[twocolumn]{article}      % Specifies the document class
\usepackage{authblk}
\usepackage{amsmath}
\usepackage{algorithm}
\usepackage{algpseudocode}
\usepackage{graphicx}
\usepackage{ragged2e}
\makeatletter
\def\BState{\State\hskip-\ALG@thistlm}
\makeatother
                             % The preamble begins here.
\title{The Largest and the Smallest Pythagorean Number from randomly generated 1000 Natural Numbers}  % Declares the document's title.
 \author{Akshay Gupta~{IIT2017505}, \hspace{6pt}Naman Deept~{IIT2017507},  \hspace{6pt}Snigdha Dobhal~{IIT2017506}
 }
\date{\today}      % Deleting this command produces today's date.
\renewcommand\Authands{ and }
\newcommand{\ip}[2]{(#1, #2)}
                             % Defines \ip{arg1}{arg2} to mean
                             % (arg1, arg2).

%\newcommand{\ip}[2]{\langle #1 | #2\rangle}
                             % This is an alternative definition of
                             % \ip that is commented out.

\begin{document}             % End of preamble and beginning of text.
\begin{titlepage}

\newcommand{\HRule}{\rule{\linewidth}{0.5mm}} % Defines a new command for the horizontal lines, change thickness here

\center % Center everything on the page
 
%----------------------------------------------------------------------------------------
%	HEADING SECTIONS
%----------------------------------------------------------------------------------------

\textsc{\LARGE Assignment 1}\\[1.5cm] % Name of your university/college
\includegraphics[scale=0.7]{iiitlogo.jpg}\\[1cm] % Include a department/university logo - this will require the graphicx package
\textsc{\Large Design and Analysis of Algorithm}\\[0.5cm] % Major heading such as course name
\textsc{\large IDAA432C}\\[0.5cm] % Minor heading such as course title

%----------------------------------------------------------------------------------------
%	TITLE SECTION
%----------------------------------------------------------------------------------------

\HRule \\[0.4cm]
{ \large \bfseries Largest and Smallest Pythagorean Numbers from a Randomly Generated Array of Natural Numbers. }\\[0.4cm] % Title of your document
\HRule \\[1.5cm]
 
%----------------------------------------------------------------------------------------
%	AUTHOR SECTION
%----------------------------------------------------------------------------------------

\begin{minipage}{0.4\textwidth}
\begin{flushleft} \large
\textbf{\emph{Submitted By:}}\\
Akshay Gupta \textbf{(IIT2017505)}\\
Naman Deept \textbf{(IIT2017507)}\\
Snigdha Dobhal \textbf{(IIT2017506)}
\end{flushleft}

\end{minipage}\\[2cm]

% If you don't want a supervisor, uncomment the two lines below and remove the section above
%\Large \emph{Author:}\\
%John \textsc{Smith}\\[3cm] % Your name

%----------------------------------------------------------------------------------------
%	DATE SECTION
%----------------------------------------------------------------------------------------

{\large \today}\\[2cm] % Date, change the \today to a set date if you want to be precise

\vfill % Fill the rest of the page with whitespace

\end{titlepage}
\maketitle                   % Produces the title.


    % Produces section heading.  Lower-level
                             % sections are begun with similar 
                             % \subsection and \subsubsection commands.

\section{Abstract} 
\textbf{T}his paper introduces several algorithms to find the largest and the smallest \textbf{Pythagorean} Numbers from the randomly generated array of \textbf{N}atural Numbers and then the most optimal approach to the problem is considered to efficiently solve the problem.\\
\textbf{Keywords}: \textit{Pythagorean}, \textit{Pythagorean Triplets}, \textit{Primes}, \textit{Largest}, \textit{Smallest}
\section{Introduction}
{ \textbf{Pythagorean Number:}} A Natural number \textbf{a} is said to be a Pythagorean number if it can be written as sum of the square of two smaller Natural numbers \textbf{b} and \textbf{c} , such that ,
\begin{center}\textbf{$a^2 = b^2 + c^2$}\end{center}
Such a triple is commonly written \textbf{(a, b, c)}, and a well-known example is (3, 4, 5). If (a, b, c) is a Pythagorean triple, then so is \textbf{(ka, kb, kc)} for any positive integer k. A primitive Pythagorean triple is one in which a, b and c are co-prime (that is, they have no common divisor larger than 1). A triangle whose sides form a Pythagorean triple is called a Pythagorean triangle, and is necessarily a right triangle. Any Pythagorean Number is a multiple of a \textbf{Pythagorean} prime \textit{(n)} which is prime as well as of the form of \textit{4n+1}.\\\ When searching for integer solutions, the equation \textbf{$a^2 + b^2 = c^2$} is a Diophantine equation. Thus Pythagorean triples are among the oldest known solutions of a nonlinear Diophantine equation.
\section{Proposed Method}
\textbf{Input}: Given 1000 randomly generated Natural numbers in an array , the aim is to find the largest and the smallest Pythagorean Number from this array of natural numbers ranging in between $10^4$ to $10^6$.
\subsection {Naive approach}
First method could be to check for all the numbers less than the value of $\textbf{n}$ for a given number $\textbf{n}$ if it can be written as sum of squares of such numbers. For this we need to find a pair of integer numbers $\textbf{i}$ and $\textbf{j}$ such that if ${i^2 + j^2=n^2}$ then we can clearly say the given number is a Pythagorean number ,else if the condition fails to hold for every possibles integers $\textbf{i}$ and $\textbf{j}$ then our number is not Pythagorean.
\subsubsection{Algorithm for the Naive Approach}
\begin{algorithm}
\begin{algorithmic}[1]
\Procedure{checkPythagorean}{$n$}
\For {$i \gets 3 \hspace{4pt} to \hspace{4pt} ${n-1}$ $}
\For{$j \gets ${i+1}$ \hspace{4pt} to \hspace{4pt} n$}
\If{ {$i^2 + j^2$ equals $n^2$}}
\State return 1
\EndIf
\EndFor
\EndFor
\State return 0
\EndProcedure
\end{algorithmic}
\end{algorithm}
\subsubsection{Time analysis}
\textbf{Best Case}: The best case complexity will hold here if the value of ${n}$ equals 5. For the value 5 both the loops will run for 1 time.Hence for the best case complexity the total time taken will be (3+3) units (1 for each loop computation) .So for the best case complexity , the time taken will be of the order of 1 ,so the time taken in this case will be ${t_\omega}$ = 6
But in the given case the numbers are in the range of ${10^4}$ to ${10^6}$ .So the complexity will be increased for sure.The best case will be computed for ${10^4}$ .Since ${10^4}$ is a Pythagorean,So we can split ${i}$ and ${j}$ such that ${i^2}$+${j^2}$=${10^8}$. So the minimum time taken in this case will be (assuming every operation takes a unit time) which is given by \begin{center} ${t_\Omega}$ $\propto\ {\sum_{i=3}^{6.{10^3}}$ $(n-i-1)$
\end{center} which approximates to $\sim \Omega(n^2)$  for the given range of n. Hence the complexity will be \Omega(${n^2}$).\\\ 
\textbf{Average Case} : The average case complexity will hold for the value where the loops are running by prerequisite greater time than the best case but lesser time than the worst case .Average case complexity is denoted by $\theta$ notation when the complexities in the best case and the worst case are the same.In this case it is same. So complexity is \theta(${n^2}$).\\\
\textbf{Worst Case} : The worst case complexity can occur if the number is not indeed a Pythagorean number .So the entire loop iteration will run for maximum times and so the complexity will be O(${n^2}$).

\subsection{Better Approach}
A better method could be to check whether for every number from \textit{i} to given number \textit{n} if \textbf{$n^2-i^2$} is a perfect square .
\newpage
\subsubsection{Algorithm for the better approach }
\begin{algorithm}
\begin{algorithmic}[1]
\Procedure{checkPythagorean}{$n$}
\For {$i \gets 3 \hspace{4pt} to \hspace{4pt} ${n}$ $}
\State $k$ \gets \hspace{2pt} $\sqrt{(n^2-i^2)}$
\State $l$ \gets \hspace{2pt} $k^2$
\If{ {$l$ equals $n^2-i^2$}}
\State return 1
\EndIf
\EndFor
\State return 0
\EndProcedure
\end{algorithmic}
\end{algorithm}
\subsubsection{Time Analysis}
\textbf{Best Case}: For the best case of the algorithm we need to consider the smallest possible value in the range of the number ,and the value is ${10^4}$ which itself takes ${6.10^3-2}$ iterations. Consider the time taken for the single iteration to be ${x}$ units .So the total time will be approximately t_\omega \propto  ${(6.10^3-2)}.x$ and hence the complexity becomes overall closer to n.So the notation will be $\Omega(${n}$)$ .\\
\textbf{Average Case} : In this case the complexity is $\Theta(n)$.\\
\textbf{Worst Case} : The worst case complexity can occur if the number is not indeed a Pythagorean number .So  the complexity will be O(${n}$).\\
\subsection{Optimal Approach}
We know that all Pythagorean number are multiples of Pythagorean prime which are prime number of the form of ${4n+1}$. Now to check whether the given number is Pythagorean number or not we can check if any of the prime factor of the given number is a Pythagorean prime or not . 
\newpage
\subsubsection{Algorithm for the Optimal Approach}
\begin{algorithm}
\begin{algorithmic}[1]
\Procedure{checkPythagorean}{$n$}
\While{2 divides n}
\State $n \gets n/2$
\EndWhile
\For{$i \gets3 \hspace{2pt} to\hspace{2pt} \sqrt{n} \hspace{1pt}$ in step of 2}
\If{i divides n  and 4 divides (n-1)}
\State return 1
\EndIf
\While{i divides n}
\State $n \gets n/i$
\EndWhile
\EndFor
\If{$n>2$ and 4 divides (n-1)}
\State return 1
\EndIf
\State return 0
\EndProcedure
\end{algorithmic}
\end{algorithm}.
\subsubsection{Time Analysis}
\textbf{Best Case}:Best case holds for the least possible time to be taken from the number in the given range.In our case consider the smallest number say ${10^4}$ in this case the complexity will be given as \begin{center}
    ${t_\omega}$  $\propto\ ${log_2{10^4}+v}$  
\end{center}where ${v}$ is a lesser quantity than ${log_2{10^4}}$ .So the complexity is ${\omega(log_2{n})}$.This complexity will hold perfect for the number which are perfect powers of 2 that is of the form ${2^n}$.\\
\textbf{Worst Case} : The worst case complexity can occur if the number is not indeed a Pythagorean number. So  the loop iteration will run for ${log_2{k}}$ + ${\sqrt{z}}$ times such that ${z=(n/2^k)}$. Out of both ${\sqrt{n}}$ is greater and so the time will heavily be determined by this factor.So the complexity here will be ${o({\sqrt{n})}}$ .\\
\begin{figure}[h!]
\includegraphics[scale=0.3]{optgraph.jpg}
\caption{Comparing best case and worst case complexity in optimal approach}
\end{figure}
\newpage
\subsection{Main Function}
\subsubsection{\textbf{V}ariables:}
\paragraph{1)}\textbf{a[]} is the given array of randomly generated natural numbers of size 1000 which is in the range from $10^4$ to $10^6$ .
\paragraph{2)}\textbf{min} is an integer to store the minimum Pythagorean number from the given array of numbers and is initialized to 0.
\paragraph{3)}\textbf{max} is an integer to store the maximum Pythagorean number from the given array of numbers and is initialized to 1000000.
\begin{algorithm}
\begin{algorithmic}
\Procedure{Main}{}
\For{$i \gets 0 \hspace{2pt} to \hspace{2pt}n $}
\If{$checkPythagorean(a[i]) equals \hspace{2pt} 1 $}
\If{$min > a[i]$}
\State $min \gets a[i]$
\EndIf
\If{$max < a[i]$}
\State $max \gets a[i]$
\EndIf
\EndIf
\EndFor
\EndProcedure
\end{algorithmic}
\end{algorithm}
\section{Experimental Results}
\includegraphics[scale=0.3]{expdaa1.jpg}
\caption{Figure 2:} Graph showing the comparisons of the experimental performance of the algorithm with the theoretical analysis .
\subsection{Complexity Analysis and Explanation}:
The overall complexity in given problem turns out to be ${o(k.\sqrt{n}/2 + k.\log(n))}$ where $k$ denotes the length of the array. In our experimental analysis,we considered an array of $k$ integers and then generated random numbers from $10^4$ to $10^6$ with the gap of $10^4$ to a specific value. The plot for the graph can be explained as follows : Consider a value from the randomly generated numbers if it is of the form of ${2^n.p}$ where $p$ is prime, Then the time taken in this case will be $k.({n + \sqrt{p}/2})$ which will be the worst case situation and the best case will be k.$\.log_2{n}}$ provided the loop terminates in the while block itself.Hence the overall complexity in the situation will lie in between those two values, that is
\begin{center}
    $k.\log_2{n}$ $<$ $t_{avg}$ $<$ ${k.\sqrt{n} +k.p}$
\end{center}
where k is the length of the array.

\section{Discussion and Future Work}
We have used the fact that all the Pythagorean numbers are multiples of Pythagorean prime , so we only checked if the smallest Pythagorean prime divides the given number . Other way could be , as we have been given array ranging from $10^4$ to $10^6$ , the largest prime factor of such number will be less than $10^3$ if such a number is not prime else we need to check the given number is prime and of the form 4n+1 . We can store all the Pythagorean primes less than $10^6$ in an array, and check the given number if divisible by any Pythagorean prime stored in this array . Only 80 such primes exist under $10^3$ , so a lot of computations will be minimized . Also one could use sieve method to save all the Pythagorean primes and then use the same method to check if the number is Pythagorean number or not . 
\section{Conclusion}
We analyzed various algorithms to find the largest and the smallest Pythagorean number and developed optimal approach to find the desired number from the randomly generated array of numbers . We also analyzed how Pythagorean prime factors were related to the problem . However , pre-computation of the problem to generate Pythagorean primes or with sieve method , the given problem may be more efficient if memory-time complexity trade-off is considered in favor of memory.  
\section{References}
\begin{enumerate}
\item IDAA432C (Design and Analysis of Algorithm) class lecture
\item Pythagorean triples reference -\\${https://en.wikipedia.org/wiki/Pythagorean_triple}$
\item Introduction to Algorithms by Cormen
\end{enumerate}
\end{document}               % End of document.


